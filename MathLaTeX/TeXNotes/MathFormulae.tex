%! Author = Awwal M\@.
%! Date = 3/4/2023

% Preamble
\documentclass[12pt]{article}

% Packages
\usepackage{amsmath}
\usepackage{IEEEtrantools}

% Document
\begin{document}

\title{The Not So Short Intro To \LaTeXe{}}
\author{Awwal~M.}
\maketitle

% Typesetting Mathematical Formulae
\section{Typesetting Mathematical Formulae}\label{sec:typesetting-math{}}

\subsection{Building Blocks}\label{subsec:building-blocks}
    \begin{description}
        \item[Vectors] \qquad
        \begin{equation*}
            \vec{a} \qquad
            \vec{AB} \qquad
            \overrightarrow{AB}
        \end{equation*}
    \end{description}

    \begin{description}
    \item[Fraction Styles] \qquad
        \begin{equation*}
            3/8 \qquad
            \frac{3}{8} \qquad
            \tfrac{3}{8}
        \end{equation*}
    \end{description}

    \begin{description}
        \item [Binomial Coefficients] \qquad Pascal's rule is
        \begin{equation*}
            \binom{n}{k} = \binom{n-1}{k} + \binom{n-1}{k-1}
        \end{equation*}
    \end{description}

    \begin{description}
        \item [Stackrel]
        % \stackrel{#1}{#2} puts the symbol given in #1 in superscript-like size
        % over #2 which is set in its usual position.
        \begin{equation*}
            f_n(x) \stackrel{*}{\approx} 1
        \end{equation*}
    \end{description}

    \begin{description}
        \item [Substack]
        % \substack allows multi-row subscript-like entries.
        \begin{equation*}
            \sum_{\substack{0<i<n \\ j \subseteq i}}^{n}
            P(i,j) = Q(i,j)
        \end{equation*}
    \end{description}

\subsection{Single Equations that are Too Long: Multiline}\label{subsec:equations-that-are-too-long}
    \begin{description}
        \item [Multiline]
        % There is also a {multiline*} variant for preventing equation number.
        \begin{multline}
            a + b + c + d + e + f + g + h + i
            \\
            = j + k + l + m + n
        \end{multline}
        \begin{equation}
            a = b + c + d + e + f
            + g + h + i + j
            + k + l + m + n + o + p
            \label{eq:equation_too_long}
        \end{equation}
        \begin{multline}
            a = b + c + d + e + f
            + g + h + i + j \\
            + k + l + m + n + o + p
        \end{multline}
    \end{description}

\subsection{Multiple Equations}\label{subsec:multiple-equations}
    \begin{description}
        \item [Grouping with \texttt{align}]
        \hfill \\ To group multiple equations: \\
        \begin{align}
            a & = b + c \\
              & = d + e
        \end{align}
        This approach failes once a single line is too long:
        \begin{align}
            a & = b + c \\
              & = d + e + f + g + h + i
              + j + k + l \nonumber \\
              & + m + n + o \\
              & = p + q + r + s
        \end{align}
    \end{description}

    \begin{description}
        \item [IEEEquation Array]
        \hfill The strength of IEEEeqnarray is the ability to specify the number of
                columns in the equation array.\ Usually, this specification will be {rCl}.
         % Package not included yet, postponing excercise for another day.
    \end{description}

\end{document}